%\documentclass[twocolumn]{article}

\documentclass{ba-kecs}
\usepackage[pdftex]{graphicx}
\usepackage{graphicx, float, amsmath,xcolor,fixltx2e}
\Huge
\numberwithin{figure}{section}
\numberwithin{equation}{section}




\newcommand{\dkepic}[2]{ %2 referes to the amount of "parameters" in this new "method"
	\begin{figure}[H] %the H signifies that the image will be put 'Here', and can only be used by using package 'float'
	\includegraphics[width=0.5\textwidth]{#1}
	\caption{#2}
	\label{#1}
	\end{figure}
}

\begin{document}

\title{Analysis of mRNA data of patients with atherosclerosis\footnote{This thesis was prepared in partial fulfillment of the requirements
 for the Degree of Bachelor of Science in Knowledge Engineering,
University of Maastricht,  supervisors: Prof. Dr. Ir. Eric Biessens, Dr. Jo\"{e}l Karel, Dr. Evgueni Smirnov, Dr. Marco Manca and Dr. Zita Soons}}
\author{Taghi Aliyev \\}
\maketitle


%---------------------------------------------------------------------------

\begin{abstract}


\end{abstract}

%---------------------------------------------------------------------------

% Sections go here
\section{Introduction}

\section{Weighted Gene Co-Expression Network Analysis}
Weighted gene co-expression network analysis is a systems biology method for describing the correlation patterns among genes across microarray samples. Weighted correlation network analysis (WGCNA) can be used for finding clusters (modules) of highly correlated genes, for summarizing such clusters using the module eigengene or an intramodular hub gene, for relating modules to one another and to external sample traits (using eigengene network methodology), and for calculating module membership measures \cite{wgcna}.
\subsection{Selection of power}
\subsection{Module Detection}
\subsection{Interconnectivity analysis}

\section{Survival Analysis}
Survival analysis examines and models the time it takes for events to occur and it typically examines the relationship of the survival distribution to covariates \cite{cox}. One application of survival analysis that will be focused on is Cox Model.
\subsection{Cox Model}
Cox model, also known as Cox proportional hazards model, is an example to survival models that are used in survival analysis of patient data. 

\section{Regression}
This section describes regression models used for further analysis of mRNA data of patients. Regression models were used in order to compute how essential are given set of attributes (i.e., if patient is smoking or not, blood pressure level, level of diabetes etc.).
\subsection{Logistic regression}
Logistic regression is a type of probabilistic statistical classification model


\section{Experiments}
\subsection{Setup of experiments}
\subsection{Results}

\section{Conclusion}

\section{Future Work}

%---------------------------------------------------------------------------


% Bibliography

\begin{thebibliography}{99}
\bibitem{ref2} Lingxue Zhang, Seyoung Kim, "Learning Gene Networks under SNP Perturbations Using eQTL Datasets" \emph{http://dx.doi.org/10.1371\%2Fjournal.pcbi.1003420}
\bibitem{wgcna} Langfelder P, Horvath S , "WGCNA: an R package for weighted correlation network analysis". \emph{BMC Bioinformatics 2008, 9:559}
\bibitem{fdr} Storey JD and Tibshirani R. , "Statistical significance for genome-wide experiments".\emph{ Proceedings of the National Academy of Sciences", 100: 9440-9445.}
\bibitem{cox} John Fox, "Cox Proportional-Hazards Regression for Survival Data", \emph{Appendix to An R and S-PLUS Companion to Applied Regression}

\end{thebibliography}

\appendix

\end{document}
