%%%%%%%%%%%%%%%%%%%%%%%%%%%%%%%%%%%%%%%%%
% Beamer Presentation
% LaTeX Template
% Version 1.0 (10/11/12)
%
% This template has been downloaded from:
% http://www.LaTeXTemplates.com
%
% License:
% CC BY-NC-SA 3.0 (http://creativecommons.org/licenses/by-nc-sa/3.0/)
%
%%%%%%%%%%%%%%%%%%%%%%%%%%%%%%%%%%%%%%%%%

%----------------------------------------------------------------------------------------
%	PACKAGES AND THEMES
%----------------------------------------------------------------------------------------

\documentclass{beamer}

\mode<presentation> {

% The Beamer class comes with a number of default slide themes
% which change the colors and layouts of slides. Below this is a list
% of all the themes, uncomment each in turn to see what they look like.

%\usetheme{default}
%\usetheme{AnnArbor}
%\usetheme{Antibes}
%\usetheme{Bergen}
%\usetheme{Berkeley}
%\usetheme{Berlin}
%\usetheme{Boadilla}
%\usetheme{CambridgeUS}
%\usetheme{Copenhagen}
%\usetheme{Darmstadt}
%\usetheme{Dresden}
%\usetheme{Frankfurt}
%\usetheme{Goettingen}
%\usetheme{Hannover}
%\usetheme{Ilmenau}
%\usetheme{JuanLesPins}
%\usetheme{Luebeck}
%\usetheme{Madrid}
%\usetheme{Malmoe}
%\usetheme{Marburg}
%\usetheme{Montpellier}
%\usetheme{PaloAlto}
\usetheme{Pittsburgh}
%\usetheme{Rochester}
%\usetheme{Singapore}
%\usetheme{Szeged}
%\usetheme{Warsaw}

% As well as themes, the Beamer class has a number of color themes
% for any slide theme. Uncomment each of these in turn to see how it
% changes the colors of your current slide theme.

%\usecolortheme{albatross}
%\usecolortheme{beaver}
%\usecolortheme{beetle}
%\usecolortheme{crane}
%\usecolortheme{dolphin}
%\usecolortheme{dove}
%\usecolortheme{fly}
%\usecolortheme{lily}
%\usecolortheme{orchid}
%\usecolortheme{rose}
%\usecolortheme{seagull}
%\usecolortheme{seahorse}
%\usecolortheme{whale}
%\usecolortheme{wolverine}

%\setbeamertemplate{footline} % To remove the footer line in all slides uncomment this line
%\setbeamertemplate{footline}[page number] % To replace the footer line in all slides with a simple slide count uncomment this line

%\setbeamertemplate{navigation symbols}{} % To remove the navigation symbols from the bottom of all slides uncomment this line
}

\usepackage{graphicx} % Allows including images
\usepackage{booktabs} % Allows the use of \toprule, \midrule and \bottomrule in tables

%----------------------------------------------------------------------------------------
%	TITLE PAGE
%----------------------------------------------------------------------------------------

\title[Short title]{Sense making of cardiovascular diseases using network analysis} % The short title appears at the bottom of every slide, the full title is only on the title page

\author{Taghi Aliyev} % Your name
\institute[MUMC+] % Your institution as it will appear on the bottom of every slide, may be shorthand to save space
{
MUMC+ \\ % Your institution for the title page
\medskip
\textit{taghialiyev@gmail.com} % Your email address
}
\date{\today} % Date, can be changed to a custom date

\begin{document}

\begin{frame}
\titlepage % Print the title page as the first slide
\end{frame}

%----------------------------------------------------------------------------------------
%	PRESENTATION SLIDES
%----------------------------------------------------------------------------------------

%------------------------------------------------
%\section{Introduction}
\begin{frame}
\frametitle{Outline}
\begin{itemize}
\item \textbf{Introduction}
	\begin{itemize}
		\item What is the topic
	\end{itemize}
	\vspace{1mm}
\item \textbf{Small recap of last time}
	\begin{itemize}
		\item Where
		\item Some facts
	\end{itemize}
	\vspace{1mm}
\item \textbf{New stuff}
	\begin{itemize}
		\item Software tools that I will use
		\item Techniques (Not completely decided, just bunch of techniques)
	\end{itemize}
\vspace{1mm}
\item \textbf{Main research questions}
\vspace{1mm}
\end{itemize}
\end{frame}

\begin{frame}
\frametitle{Introduction}
\begin{itemize}
\item \textbf{What} is the topic?
	\begin{itemize}
		\item Research on network analysis of cardiovascular diseases
		\item Array data (I will show example)
		\item Computational problem because of the data size (n = 500, 10-15 thousand attributes)
	\end{itemize}
	\vspace{1mm}
\item \textbf{Why} is this important?
	\begin{itemize}
		\item People die and we do not want that
	\end{itemize}
\end{itemize}
\end{frame}

\begin{frame}
\frametitle{Small Recap}
\begin{itemize}
	\item \textbf{Where?}
	\begin{itemize}
		\item MUMC+(Maastricht Hospital) + supervision from CERN(with chance of possible visit)
	\end{itemize}
	\item \textbf{No heartbeat = Dead(returns true in 100 percent of cases)}
	\item \textbf{Atherosclerosis}
	\begin{itemize}
		\item Is a disease in which plaque builds up inside your arteries
		\item Leads to heart attacks, stroke and sometimes death
		\item Is one of the main causes for death of people
	\end{itemize}
\end{itemize}
\end{frame}

\begin{frame}
\frametitle{New stuff}
\begin{itemize}
\item \textbf{Software tools}
	\begin{itemize}
		\item R
		\item WEKA
		\item Java
		\item Cytoscape(Optional, mainly for visualization purposes)
		\item GLPK
	\end{itemize}
\item \textbf{Possible techniques}
	\begin{itemize}
		\item WGCNA(Weighted Gene Co-Expression Network Analysis)
		\item Artificial Neural Networks
		\item Network Simplex Method (Min cut problem)
		\item Bunch of other data mining/machine learning techniques
		\begin{itemize}
			\item Clusterings, SVM, Regression(Multivariate) etc.
		\end{itemize}
	\end{itemize}
\end{itemize}
\end{frame}

\begin{frame}
\frametitle{Main research questions}
	\begin{itemize}
		\item Can we model how human genes react to cardiovascular diseases?
		\item Can we predict heart attack, stroke for some given patient given information about patient?
		\item Can we find regulatory genes concerning this disease?
		\item Main of all : Can we cure it? Or decrease it to almost non-existing amount?
	\end{itemize}
\end{frame}

%------------------------------------------------

\begin{frame}
\frametitle{Thanks}
	\begin{itemize}
		\item Thanks for listening and if there are any questions I would like to give a try at answering them :)
	\end{itemize}
\end{frame}

%----------------------------------------------------------------------------------------

\end{document}